
% Default to the notebook output style

    


% Inherit from the specified cell style.




    
\documentclass[11pt]{article}
    \usepackage{fontspec, xunicode, xltxtra}

    \setmainfont{Microsoft YaHei}

    \usepackage{ctex} 
    
    
    \usepackage[T1]{fontenc}
    % Nicer default font (+ math font) than Computer Modern for most use cases
    \usepackage{mathpazo}

    % Basic figure setup, for now with no caption control since it's done
    % automatically by Pandoc (which extracts ![](path) syntax from Markdown).
    \usepackage{graphicx}
    % We will generate all images so they have a width \maxwidth. This means
    % that they will get their normal width if they fit onto the page, but
    % are scaled down if they would overflow the margins.
    \makeatletter
    \def\maxwidth{\ifdim\Gin@nat@width>\linewidth\linewidth
    \else\Gin@nat@width\fi}
    \makeatother
    \let\Oldincludegraphics\includegraphics
    % Set max figure width to be 80% of text width, for now hardcoded.
    \renewcommand{\includegraphics}[1]{\Oldincludegraphics[width=.8\maxwidth]{#1}}
    % Ensure that by default, figures have no caption (until we provide a
    % proper Figure object with a Caption API and a way to capture that
    % in the conversion process - todo).
    \usepackage{caption}
    \DeclareCaptionLabelFormat{nolabel}{}
    \captionsetup{labelformat=nolabel}

    \usepackage{adjustbox} % Used to constrain images to a maximum size 
    \usepackage{xcolor} % Allow colors to be defined
    \usepackage{enumerate} % Needed for markdown enumerations to work
    \usepackage{geometry} % Used to adjust the document margins
    \usepackage{amsmath} % Equations
    \usepackage{amssymb} % Equations
    \usepackage{textcomp} % defines textquotesingle
    % Hack from http://tex.stackexchange.com/a/47451/13684:
    \AtBeginDocument{%
        \def\PYZsq{\textquotesingle}% Upright quotes in Pygmentized code
    }
    \usepackage{upquote} % Upright quotes for verbatim code
    \usepackage{eurosym} % defines \euro
    \usepackage[mathletters]{ucs} % Extended unicode (utf-8) support
    \usepackage[utf8x]{inputenc} % Allow utf-8 characters in the tex document
    \usepackage{fancyvrb} % verbatim replacement that allows latex
    \usepackage{grffile} % extends the file name processing of package graphics 
                         % to support a larger range 
    % The hyperref package gives us a pdf with properly built
    % internal navigation ('pdf bookmarks' for the table of contents,
    % internal cross-reference links, web links for URLs, etc.)
    \usepackage{hyperref}
    \usepackage{longtable} % longtable support required by pandoc >1.10
    \usepackage{booktabs}  % table support for pandoc > 1.12.2
    \usepackage[inline]{enumitem} % IRkernel/repr support (it uses the enumerate* environment)
    \usepackage[normalem]{ulem} % ulem is needed to support strikethroughs (\sout)
                                % normalem makes italics be italics, not underlines
    

    
    
    % Colors for the hyperref package
    \definecolor{urlcolor}{rgb}{0,.145,.698}
    \definecolor{linkcolor}{rgb}{.71,0.21,0.01}
    \definecolor{citecolor}{rgb}{.12,.54,.11}

    % ANSI colors
    \definecolor{ansi-black}{HTML}{3E424D}
    \definecolor{ansi-black-intense}{HTML}{282C36}
    \definecolor{ansi-red}{HTML}{E75C58}
    \definecolor{ansi-red-intense}{HTML}{B22B31}
    \definecolor{ansi-green}{HTML}{00A250}
    \definecolor{ansi-green-intense}{HTML}{007427}
    \definecolor{ansi-yellow}{HTML}{DDB62B}
    \definecolor{ansi-yellow-intense}{HTML}{B27D12}
    \definecolor{ansi-blue}{HTML}{208FFB}
    \definecolor{ansi-blue-intense}{HTML}{0065CA}
    \definecolor{ansi-magenta}{HTML}{D160C4}
    \definecolor{ansi-magenta-intense}{HTML}{A03196}
    \definecolor{ansi-cyan}{HTML}{60C6C8}
    \definecolor{ansi-cyan-intense}{HTML}{258F8F}
    \definecolor{ansi-white}{HTML}{C5C1B4}
    \definecolor{ansi-white-intense}{HTML}{A1A6B2}

    % commands and environments needed by pandoc snippets
    % extracted from the output of `pandoc -s`
    \providecommand{\tightlist}{%
      \setlength{\itemsep}{0pt}\setlength{\parskip}{0pt}}
    \DefineVerbatimEnvironment{Highlighting}{Verbatim}{commandchars=\\\{\}}
    % Add ',fontsize=\small' for more characters per line
    \newenvironment{Shaded}{}{}
    \newcommand{\KeywordTok}[1]{\textcolor[rgb]{0.00,0.44,0.13}{\textbf{{#1}}}}
    \newcommand{\DataTypeTok}[1]{\textcolor[rgb]{0.56,0.13,0.00}{{#1}}}
    \newcommand{\DecValTok}[1]{\textcolor[rgb]{0.25,0.63,0.44}{{#1}}}
    \newcommand{\BaseNTok}[1]{\textcolor[rgb]{0.25,0.63,0.44}{{#1}}}
    \newcommand{\FloatTok}[1]{\textcolor[rgb]{0.25,0.63,0.44}{{#1}}}
    \newcommand{\CharTok}[1]{\textcolor[rgb]{0.25,0.44,0.63}{{#1}}}
    \newcommand{\StringTok}[1]{\textcolor[rgb]{0.25,0.44,0.63}{{#1}}}
    \newcommand{\CommentTok}[1]{\textcolor[rgb]{0.38,0.63,0.69}{\textit{{#1}}}}
    \newcommand{\OtherTok}[1]{\textcolor[rgb]{0.00,0.44,0.13}{{#1}}}
    \newcommand{\AlertTok}[1]{\textcolor[rgb]{1.00,0.00,0.00}{\textbf{{#1}}}}
    \newcommand{\FunctionTok}[1]{\textcolor[rgb]{0.02,0.16,0.49}{{#1}}}
    \newcommand{\RegionMarkerTok}[1]{{#1}}
    \newcommand{\ErrorTok}[1]{\textcolor[rgb]{1.00,0.00,0.00}{\textbf{{#1}}}}
    \newcommand{\NormalTok}[1]{{#1}}
    
    % Additional commands for more recent versions of Pandoc
    \newcommand{\ConstantTok}[1]{\textcolor[rgb]{0.53,0.00,0.00}{{#1}}}
    \newcommand{\SpecialCharTok}[1]{\textcolor[rgb]{0.25,0.44,0.63}{{#1}}}
    \newcommand{\VerbatimStringTok}[1]{\textcolor[rgb]{0.25,0.44,0.63}{{#1}}}
    \newcommand{\SpecialStringTok}[1]{\textcolor[rgb]{0.73,0.40,0.53}{{#1}}}
    \newcommand{\ImportTok}[1]{{#1}}
    \newcommand{\DocumentationTok}[1]{\textcolor[rgb]{0.73,0.13,0.13}{\textit{{#1}}}}
    \newcommand{\AnnotationTok}[1]{\textcolor[rgb]{0.38,0.63,0.69}{\textbf{\textit{{#1}}}}}
    \newcommand{\CommentVarTok}[1]{\textcolor[rgb]{0.38,0.63,0.69}{\textbf{\textit{{#1}}}}}
    \newcommand{\VariableTok}[1]{\textcolor[rgb]{0.10,0.09,0.49}{{#1}}}
    \newcommand{\ControlFlowTok}[1]{\textcolor[rgb]{0.00,0.44,0.13}{\textbf{{#1}}}}
    \newcommand{\OperatorTok}[1]{\textcolor[rgb]{0.40,0.40,0.40}{{#1}}}
    \newcommand{\BuiltInTok}[1]{{#1}}
    \newcommand{\ExtensionTok}[1]{{#1}}
    \newcommand{\PreprocessorTok}[1]{\textcolor[rgb]{0.74,0.48,0.00}{{#1}}}
    \newcommand{\AttributeTok}[1]{\textcolor[rgb]{0.49,0.56,0.16}{{#1}}}
    \newcommand{\InformationTok}[1]{\textcolor[rgb]{0.38,0.63,0.69}{\textbf{\textit{{#1}}}}}
    \newcommand{\WarningTok}[1]{\textcolor[rgb]{0.38,0.63,0.69}{\textbf{\textit{{#1}}}}}
    
    
    % Define a nice break command that doesn't care if a line doesn't already
    % exist.
    \def\br{\hspace*{\fill} \\* }
    % Math Jax compatability definitions
    \def\gt{>}
    \def\lt{<}
    % Document parameters
    \title{第三次金融工程学作业}
    \author{\heiti \zihao{5}杨宸宇\\\heiti \zihao{5}2016301550186}
    \date{}
    
    
    

    % Pygments definitions
    
\makeatletter
\def\PY@reset{\let\PY@it=\relax \let\PY@bf=\relax%
    \let\PY@ul=\relax \let\PY@tc=\relax%
    \let\PY@bc=\relax \let\PY@ff=\relax}
\def\PY@tok#1{\csname PY@tok@#1\endcsname}
\def\PY@toks#1+{\ifx\relax#1\empty\else%
    \PY@tok{#1}\expandafter\PY@toks\fi}
\def\PY@do#1{\PY@bc{\PY@tc{\PY@ul{%
    \PY@it{\PY@bf{\PY@ff{#1}}}}}}}
\def\PY#1#2{\PY@reset\PY@toks#1+\relax+\PY@do{#2}}

\expandafter\def\csname PY@tok@w\endcsname{\def\PY@tc##1{\textcolor[rgb]{0.73,0.73,0.73}{##1}}}
\expandafter\def\csname PY@tok@c\endcsname{\let\PY@it=\textit\def\PY@tc##1{\textcolor[rgb]{0.25,0.50,0.50}{##1}}}
\expandafter\def\csname PY@tok@cp\endcsname{\def\PY@tc##1{\textcolor[rgb]{0.74,0.48,0.00}{##1}}}
\expandafter\def\csname PY@tok@k\endcsname{\let\PY@bf=\textbf\def\PY@tc##1{\textcolor[rgb]{0.00,0.50,0.00}{##1}}}
\expandafter\def\csname PY@tok@kp\endcsname{\def\PY@tc##1{\textcolor[rgb]{0.00,0.50,0.00}{##1}}}
\expandafter\def\csname PY@tok@kt\endcsname{\def\PY@tc##1{\textcolor[rgb]{0.69,0.00,0.25}{##1}}}
\expandafter\def\csname PY@tok@o\endcsname{\def\PY@tc##1{\textcolor[rgb]{0.40,0.40,0.40}{##1}}}
\expandafter\def\csname PY@tok@ow\endcsname{\let\PY@bf=\textbf\def\PY@tc##1{\textcolor[rgb]{0.67,0.13,1.00}{##1}}}
\expandafter\def\csname PY@tok@nb\endcsname{\def\PY@tc##1{\textcolor[rgb]{0.00,0.50,0.00}{##1}}}
\expandafter\def\csname PY@tok@nf\endcsname{\def\PY@tc##1{\textcolor[rgb]{0.00,0.00,1.00}{##1}}}
\expandafter\def\csname PY@tok@nc\endcsname{\let\PY@bf=\textbf\def\PY@tc##1{\textcolor[rgb]{0.00,0.00,1.00}{##1}}}
\expandafter\def\csname PY@tok@nn\endcsname{\let\PY@bf=\textbf\def\PY@tc##1{\textcolor[rgb]{0.00,0.00,1.00}{##1}}}
\expandafter\def\csname PY@tok@ne\endcsname{\let\PY@bf=\textbf\def\PY@tc##1{\textcolor[rgb]{0.82,0.25,0.23}{##1}}}
\expandafter\def\csname PY@tok@nv\endcsname{\def\PY@tc##1{\textcolor[rgb]{0.10,0.09,0.49}{##1}}}
\expandafter\def\csname PY@tok@no\endcsname{\def\PY@tc##1{\textcolor[rgb]{0.53,0.00,0.00}{##1}}}
\expandafter\def\csname PY@tok@nl\endcsname{\def\PY@tc##1{\textcolor[rgb]{0.63,0.63,0.00}{##1}}}
\expandafter\def\csname PY@tok@ni\endcsname{\let\PY@bf=\textbf\def\PY@tc##1{\textcolor[rgb]{0.60,0.60,0.60}{##1}}}
\expandafter\def\csname PY@tok@na\endcsname{\def\PY@tc##1{\textcolor[rgb]{0.49,0.56,0.16}{##1}}}
\expandafter\def\csname PY@tok@nt\endcsname{\let\PY@bf=\textbf\def\PY@tc##1{\textcolor[rgb]{0.00,0.50,0.00}{##1}}}
\expandafter\def\csname PY@tok@nd\endcsname{\def\PY@tc##1{\textcolor[rgb]{0.67,0.13,1.00}{##1}}}
\expandafter\def\csname PY@tok@s\endcsname{\def\PY@tc##1{\textcolor[rgb]{0.73,0.13,0.13}{##1}}}
\expandafter\def\csname PY@tok@sd\endcsname{\let\PY@it=\textit\def\PY@tc##1{\textcolor[rgb]{0.73,0.13,0.13}{##1}}}
\expandafter\def\csname PY@tok@si\endcsname{\let\PY@bf=\textbf\def\PY@tc##1{\textcolor[rgb]{0.73,0.40,0.53}{##1}}}
\expandafter\def\csname PY@tok@se\endcsname{\let\PY@bf=\textbf\def\PY@tc##1{\textcolor[rgb]{0.73,0.40,0.13}{##1}}}
\expandafter\def\csname PY@tok@sr\endcsname{\def\PY@tc##1{\textcolor[rgb]{0.73,0.40,0.53}{##1}}}
\expandafter\def\csname PY@tok@ss\endcsname{\def\PY@tc##1{\textcolor[rgb]{0.10,0.09,0.49}{##1}}}
\expandafter\def\csname PY@tok@sx\endcsname{\def\PY@tc##1{\textcolor[rgb]{0.00,0.50,0.00}{##1}}}
\expandafter\def\csname PY@tok@m\endcsname{\def\PY@tc##1{\textcolor[rgb]{0.40,0.40,0.40}{##1}}}
\expandafter\def\csname PY@tok@gh\endcsname{\let\PY@bf=\textbf\def\PY@tc##1{\textcolor[rgb]{0.00,0.00,0.50}{##1}}}
\expandafter\def\csname PY@tok@gu\endcsname{\let\PY@bf=\textbf\def\PY@tc##1{\textcolor[rgb]{0.50,0.00,0.50}{##1}}}
\expandafter\def\csname PY@tok@gd\endcsname{\def\PY@tc##1{\textcolor[rgb]{0.63,0.00,0.00}{##1}}}
\expandafter\def\csname PY@tok@gi\endcsname{\def\PY@tc##1{\textcolor[rgb]{0.00,0.63,0.00}{##1}}}
\expandafter\def\csname PY@tok@gr\endcsname{\def\PY@tc##1{\textcolor[rgb]{1.00,0.00,0.00}{##1}}}
\expandafter\def\csname PY@tok@ge\endcsname{\let\PY@it=\textit}
\expandafter\def\csname PY@tok@gs\endcsname{\let\PY@bf=\textbf}
\expandafter\def\csname PY@tok@gp\endcsname{\let\PY@bf=\textbf\def\PY@tc##1{\textcolor[rgb]{0.00,0.00,0.50}{##1}}}
\expandafter\def\csname PY@tok@go\endcsname{\def\PY@tc##1{\textcolor[rgb]{0.53,0.53,0.53}{##1}}}
\expandafter\def\csname PY@tok@gt\endcsname{\def\PY@tc##1{\textcolor[rgb]{0.00,0.27,0.87}{##1}}}
\expandafter\def\csname PY@tok@err\endcsname{\def\PY@bc##1{\setlength{\fboxsep}{0pt}\fcolorbox[rgb]{1.00,0.00,0.00}{1,1,1}{\strut ##1}}}
\expandafter\def\csname PY@tok@kc\endcsname{\let\PY@bf=\textbf\def\PY@tc##1{\textcolor[rgb]{0.00,0.50,0.00}{##1}}}
\expandafter\def\csname PY@tok@kd\endcsname{\let\PY@bf=\textbf\def\PY@tc##1{\textcolor[rgb]{0.00,0.50,0.00}{##1}}}
\expandafter\def\csname PY@tok@kn\endcsname{\let\PY@bf=\textbf\def\PY@tc##1{\textcolor[rgb]{0.00,0.50,0.00}{##1}}}
\expandafter\def\csname PY@tok@kr\endcsname{\let\PY@bf=\textbf\def\PY@tc##1{\textcolor[rgb]{0.00,0.50,0.00}{##1}}}
\expandafter\def\csname PY@tok@bp\endcsname{\def\PY@tc##1{\textcolor[rgb]{0.00,0.50,0.00}{##1}}}
\expandafter\def\csname PY@tok@fm\endcsname{\def\PY@tc##1{\textcolor[rgb]{0.00,0.00,1.00}{##1}}}
\expandafter\def\csname PY@tok@vc\endcsname{\def\PY@tc##1{\textcolor[rgb]{0.10,0.09,0.49}{##1}}}
\expandafter\def\csname PY@tok@vg\endcsname{\def\PY@tc##1{\textcolor[rgb]{0.10,0.09,0.49}{##1}}}
\expandafter\def\csname PY@tok@vi\endcsname{\def\PY@tc##1{\textcolor[rgb]{0.10,0.09,0.49}{##1}}}
\expandafter\def\csname PY@tok@vm\endcsname{\def\PY@tc##1{\textcolor[rgb]{0.10,0.09,0.49}{##1}}}
\expandafter\def\csname PY@tok@sa\endcsname{\def\PY@tc##1{\textcolor[rgb]{0.73,0.13,0.13}{##1}}}
\expandafter\def\csname PY@tok@sb\endcsname{\def\PY@tc##1{\textcolor[rgb]{0.73,0.13,0.13}{##1}}}
\expandafter\def\csname PY@tok@sc\endcsname{\def\PY@tc##1{\textcolor[rgb]{0.73,0.13,0.13}{##1}}}
\expandafter\def\csname PY@tok@dl\endcsname{\def\PY@tc##1{\textcolor[rgb]{0.73,0.13,0.13}{##1}}}
\expandafter\def\csname PY@tok@s2\endcsname{\def\PY@tc##1{\textcolor[rgb]{0.73,0.13,0.13}{##1}}}
\expandafter\def\csname PY@tok@sh\endcsname{\def\PY@tc##1{\textcolor[rgb]{0.73,0.13,0.13}{##1}}}
\expandafter\def\csname PY@tok@s1\endcsname{\def\PY@tc##1{\textcolor[rgb]{0.73,0.13,0.13}{##1}}}
\expandafter\def\csname PY@tok@mb\endcsname{\def\PY@tc##1{\textcolor[rgb]{0.40,0.40,0.40}{##1}}}
\expandafter\def\csname PY@tok@mf\endcsname{\def\PY@tc##1{\textcolor[rgb]{0.40,0.40,0.40}{##1}}}
\expandafter\def\csname PY@tok@mh\endcsname{\def\PY@tc##1{\textcolor[rgb]{0.40,0.40,0.40}{##1}}}
\expandafter\def\csname PY@tok@mi\endcsname{\def\PY@tc##1{\textcolor[rgb]{0.40,0.40,0.40}{##1}}}
\expandafter\def\csname PY@tok@il\endcsname{\def\PY@tc##1{\textcolor[rgb]{0.40,0.40,0.40}{##1}}}
\expandafter\def\csname PY@tok@mo\endcsname{\def\PY@tc##1{\textcolor[rgb]{0.40,0.40,0.40}{##1}}}
\expandafter\def\csname PY@tok@ch\endcsname{\let\PY@it=\textit\def\PY@tc##1{\textcolor[rgb]{0.25,0.50,0.50}{##1}}}
\expandafter\def\csname PY@tok@cm\endcsname{\let\PY@it=\textit\def\PY@tc##1{\textcolor[rgb]{0.25,0.50,0.50}{##1}}}
\expandafter\def\csname PY@tok@cpf\endcsname{\let\PY@it=\textit\def\PY@tc##1{\textcolor[rgb]{0.25,0.50,0.50}{##1}}}
\expandafter\def\csname PY@tok@c1\endcsname{\let\PY@it=\textit\def\PY@tc##1{\textcolor[rgb]{0.25,0.50,0.50}{##1}}}
\expandafter\def\csname PY@tok@cs\endcsname{\let\PY@it=\textit\def\PY@tc##1{\textcolor[rgb]{0.25,0.50,0.50}{##1}}}

\def\PYZbs{\char`\\}
\def\PYZus{\char`\_}
\def\PYZob{\char`\{}
\def\PYZcb{\char`\}}
\def\PYZca{\char`\^}
\def\PYZam{\char`\&}
\def\PYZlt{\char`\<}
\def\PYZgt{\char`\>}
\def\PYZsh{\char`\#}
\def\PYZpc{\char`\%}
\def\PYZdl{\char`\$}
\def\PYZhy{\char`\-}
\def\PYZsq{\char`\'}
\def\PYZdq{\char`\"}
\def\PYZti{\char`\~}
% for compatibility with earlier versions
\def\PYZat{@}
\def\PYZlb{[}
\def\PYZrb{]}
\makeatother


    % Exact colors from NB
    \definecolor{incolor}{rgb}{0.0, 0.0, 0.5}
    \definecolor{outcolor}{rgb}{0.545, 0.0, 0.0}



    
    % Prevent overflowing lines due to hard-to-break entities
    \sloppy 
    % Setup hyperref package
    \hypersetup{
      breaklinks=true,  % so long urls are correctly broken across lines
      colorlinks=true,
      urlcolor=urlcolor,
      linkcolor=linkcolor,
      citecolor=citecolor,
      }
    % Slightly bigger margins than the latex defaults
    
    \geometry{verbose,tmargin=1in,bmargin=1in,lmargin=1in,rmargin=1in}
    
    

    \begin{document}
    
    
    \maketitle
    
    

    
    \section{准备工作}\label{ux51c6ux5907ux5de5ux4f5c}

    \begin{Verbatim}[commandchars=\\\{\}]
{\color{incolor}In [{\color{incolor}1}]:} \PY{c+c1}{\PYZsh{} \PYZhy{}*\PYZhy{}coding:utf\PYZhy{}8\PYZhy{}*\PYZhy{}}
        \PY{c+c1}{\PYZsh{} 调用所需要的库}
        \PY{k+kn}{import} \PY{n+nn}{numpy} \PY{k}{as} \PY{n+nn}{np}
        \PY{k+kn}{import} \PY{n+nn}{pandas} \PY{k}{as} \PY{n+nn}{pd}
        \PY{k+kn}{import} \PY{n+nn}{matplotlib}\PY{n+nn}{.}\PY{n+nn}{pyplot} \PY{k}{as} \PY{n+nn}{plt}
        \PY{k+kn}{from} \PY{n+nn}{sympy} \PY{k}{import} \PY{o}{*}
        \PY{k+kn}{import} \PY{n+nn}{seaborn} \PY{k}{as} \PY{n+nn}{sns}
        \PY{n}{sns}\PY{o}{.}\PY{n}{set}\PY{p}{(}\PY{p}{)}
\end{Verbatim}


    \begin{Verbatim}[commandchars=\\\{\}]
{\color{incolor}In [{\color{incolor}2}]:} \PY{c+c1}{\PYZsh{} 更改中文字体}
        \PY{c+c1}{\PYZsh{} plt.rc(‘font’,family=[‘Microsoft YaHei’])}
\end{Verbatim}


    \section{等额本金法}\label{ux7b49ux989dux672cux91d1ux6cd5}

    \subsection{等额本金法计算公式}\label{ux7b49ux989dux672cux91d1ux6cd5ux8ba1ux7b97ux516cux5f0f}

\begin{itemize}
\tightlist
\item
  设贷款总额为A元,贷款年利率为R,月利率为i(i=R/12),贷款期限为n年,总期数为m个月(m=n*12)
\item
  每月偿还本金 = A/m
\item
  当月偿还利息 = 当月本金 * i
\item
  当月本金 = 上月本金 - 上月偿还本金
\item
  当月偿还本利和 = 当月偿还本金 + 当月偿还利息
\item
  因此进行简单的循环即可得出所有的月付本息
\end{itemize}

    \begin{Verbatim}[commandchars=\\\{\}]
{\color{incolor}In [{\color{incolor}3}]:} \PY{k}{def} \PY{n+nf}{same\PYZus{}capital}\PY{p}{(}\PY{n}{K}\PY{p}{,} \PY{n}{r}\PY{p}{,} \PY{n}{n}\PY{p}{)}\PY{p}{:}
            \PY{l+s+sd}{\PYZsq{}\PYZsq{}\PYZsq{}}
        \PY{l+s+sd}{    参数 K:本金}
        \PY{l+s+sd}{    参数 r:年利率}
        \PY{l+s+sd}{    参数 n:贷款年数}
        \PY{l+s+sd}{    输出 x:包含了月份,每月本金,每月偿还利息,每月偿还总本息和的数据集}
        \PY{l+s+sd}{    \PYZsq{}\PYZsq{}\PYZsq{}}
            \PY{c+c1}{\PYZsh{} 计算月利率}
            \PY{n}{i} \PY{o}{=} \PY{n}{r} \PY{o}{/} \PY{l+m+mi}{12}
            \PY{c+c1}{\PYZsh{} 计算总的贷款月份}
            \PY{n}{m} \PY{o}{=} \PY{l+m+mi}{12} \PY{o}{*} \PY{n}{n}
            \PY{c+c1}{\PYZsh{} 计算每月偿还本金额}
            \PY{n}{A} \PY{o}{=} \PY{n}{K} \PY{o}{/} \PY{n}{m}
            \PY{c+c1}{\PYZsh{} 构建一个空的DataFrame用于储存数据}
            \PY{n}{x} \PY{o}{=} \PY{n}{pd}\PY{o}{.}\PY{n}{DataFrame}\PY{p}{(}\PY{p}{)}
            \PY{n}{x}\PY{p}{[}\PY{l+s+s1}{\PYZsq{}}\PY{l+s+s1}{月份}\PY{l+s+s1}{\PYZsq{}}\PY{p}{]} \PY{o}{=} \PY{p}{[}\PY{n}{j} \PY{k}{for} \PY{n}{j} \PY{o+ow}{in} \PY{n+nb}{range}\PY{p}{(}\PY{l+m+mi}{1}\PY{p}{,} \PY{n}{m} \PY{o}{+} \PY{l+m+mi}{1}\PY{p}{)}\PY{p}{]}
            \PY{n}{x}\PY{p}{[}\PY{l+s+s1}{\PYZsq{}}\PY{l+s+s1}{当月本金}\PY{l+s+s1}{\PYZsq{}}\PY{p}{]} \PY{o}{=} \PY{p}{[}\PY{n}{K} \PY{o}{\PYZhy{}} \PY{n}{A} \PY{o}{*} \PY{n}{j} \PY{k}{for} \PY{n}{j} \PY{o+ow}{in} \PY{n+nb}{range}\PY{p}{(}\PY{n}{m}\PY{p}{)}\PY{p}{]}
            \PY{n}{x}\PY{p}{[}\PY{l+s+s1}{\PYZsq{}}\PY{l+s+s1}{当月应付利息}\PY{l+s+s1}{\PYZsq{}}\PY{p}{]} \PY{o}{=} \PY{n}{np}\PY{o}{.}\PY{n}{nan}
            \PY{n}{x}\PY{p}{[}\PY{l+s+s1}{\PYZsq{}}\PY{l+s+s1}{当月应付本息和}\PY{l+s+s1}{\PYZsq{}}\PY{p}{]} \PY{o}{=} \PY{n}{np}\PY{o}{.}\PY{n}{nan}
            \PY{c+c1}{\PYZsh{} 为了循环的方便先填好第一行}
            \PY{n}{x}\PY{o}{.}\PY{n}{iloc}\PY{p}{[}\PY{l+m+mi}{0}\PY{p}{,} \PY{l+m+mi}{2}\PY{p}{]} \PY{o}{=} \PY{n}{K} \PY{o}{*} \PY{n}{i}
            \PY{n}{x}\PY{o}{.}\PY{n}{iloc}\PY{p}{[}\PY{l+m+mi}{0}\PY{p}{,} \PY{l+m+mi}{3}\PY{p}{]} \PY{o}{=} \PY{n}{A} \PY{o}{+} \PY{n}{K} \PY{o}{*} \PY{n}{i}
            \PY{c+c1}{\PYZsh{} 循环重复计算每月的利息和本息和}
            \PY{k}{for} \PY{n}{j} \PY{o+ow}{in} \PY{n+nb}{range}\PY{p}{(}\PY{l+m+mi}{1}\PY{p}{,} \PY{n}{m}\PY{p}{)}\PY{p}{:}
                \PY{n}{x}\PY{o}{.}\PY{n}{iloc}\PY{p}{[}\PY{n}{j}\PY{p}{,} \PY{l+m+mi}{2}\PY{p}{]} \PY{o}{=} \PY{n}{x}\PY{o}{.}\PY{n}{iloc}\PY{p}{[}\PY{n}{j}\PY{p}{,} \PY{l+m+mi}{1}\PY{p}{]} \PY{o}{*} \PY{n}{i}
                \PY{n}{x}\PY{o}{.}\PY{n}{iloc}\PY{p}{[}\PY{n}{j}\PY{p}{,} \PY{l+m+mi}{3}\PY{p}{]} \PY{o}{=} \PY{n}{A} \PY{o}{+} \PY{n}{x}\PY{o}{.}\PY{n}{iloc}\PY{p}{[}\PY{n}{j}\PY{p}{,} \PY{l+m+mi}{2}\PY{p}{]}
            \PY{k}{return} \PY{n}{x}
\end{Verbatim}


    \subsection{等额本金法计算例子}\label{ux7b49ux989dux672cux91d1ux6cd5ux8ba1ux7b97ux4f8bux5b50}

    \begin{Verbatim}[commandchars=\\\{\}]
{\color{incolor}In [{\color{incolor}4}]:} \PY{n}{data} \PY{o}{=} \PY{n}{same\PYZus{}capital}\PY{p}{(}\PY{l+m+mi}{300000}\PY{p}{,} \PY{l+m+mf}{0.05}\PY{p}{,} \PY{l+m+mi}{15}\PY{p}{)}
        \PY{n}{plt}\PY{o}{.}\PY{n}{plot}\PY{p}{(}\PY{n}{data}\PY{p}{[}\PY{l+s+s1}{\PYZsq{}}\PY{l+s+s1}{月份}\PY{l+s+s1}{\PYZsq{}}\PY{p}{]}\PY{p}{,} \PY{n}{data}\PY{p}{[}\PY{l+s+s1}{\PYZsq{}}\PY{l+s+s1}{当月应付利息}\PY{l+s+s1}{\PYZsq{}}\PY{p}{]}\PY{p}{,} \PY{n}{color}\PY{o}{=}\PY{l+s+s1}{\PYZsq{}}\PY{l+s+s1}{blue}\PY{l+s+s1}{\PYZsq{}}\PY{p}{,} \PY{n}{linewidth}\PY{o}{=}\PY{l+m+mf}{2.5}\PY{p}{)}
        \PY{c+c1}{\PYZsh{} 添加标题}
        \PY{n}{plt}\PY{o}{.}\PY{n}{xlabel}\PY{p}{(}\PY{l+s+s1}{\PYZsq{}}\PY{l+s+s1}{month}\PY{l+s+s1}{\PYZsq{}}\PY{p}{)}
        \PY{n}{plt}\PY{o}{.}\PY{n}{ylabel}\PY{p}{(}\PY{l+s+s1}{\PYZsq{}}\PY{l+s+s1}{interest}\PY{l+s+s1}{\PYZsq{}}\PY{p}{)}
        \PY{n}{plt}\PY{o}{.}\PY{n}{title}\PY{p}{(}\PY{l+s+s1}{\PYZsq{}}\PY{l+s+s1}{Monthly Interest Pay}\PY{l+s+s1}{\PYZsq{}}\PY{p}{)}
        \PY{n}{plt}\PY{o}{.}\PY{n}{show}\PY{p}{(}\PY{p}{)}
\end{Verbatim}


    \begin{center}
    \adjustimage{max size={0.9\linewidth}{0.9\paperheight}}{output_7_0.png}
    \end{center}
    { \hspace*{\fill} \\}
    
    可以看出每月还款的利息呈现下降趋势

    \section{等额本息法}\label{ux7b49ux989dux672cux606fux6cd5}

    \subsection{等额本息法计算方法}\label{ux7b49ux989dux672cux606fux6cd5ux8ba1ux7b97ux65b9ux6cd5}

\begin{itemize}
\tightlist
\item
  首先我们可以把它看成是一种息票债券,现值为A,每月偿还x,一共偿还m期,月内部收益率等于月利率i
\item
  所以先求出月偿还总额:$x = \frac{Ai(1+i)^m}{(1+i)^m-1} $
\item
  将每月的还款日当作月末,则:
\item
  第一个月月初,本金为A,不需要还款
\item
  第一个月的还款日(当作月末),本金A的本利和为A(1+i),还款X后,下个月的本金为:A(1+i)-X
\item
  第二个月月末,A(1+i)-X的本金的本利和为:(A(1+i)-X)(1+i),还款X后,下个月的本金为:(A(1+i)-X)(1+i)-X
  = A(1+i)2-X{[}1+(1+i){]}
\item
  可以进行等差数列求和,我们可以得到第k月的本金为:\(A(1+i)^{(k-1)}-\frac{x[(1+i)^{k-1}-1]}{i}\)
\item
  而第k月的应付的利息就为本金乘利率i:\([A(1+i)^{(k-1)}-\frac{x[(1+i)^{k-1}-1]}{i}]*i\)
\item
  接着进行简单的循环即可得到债券每月的本金和利息变化情况
\end{itemize}

    \begin{Verbatim}[commandchars=\\\{\}]
{\color{incolor}In [{\color{incolor}5}]:} \PY{k}{def} \PY{n+nf}{same\PYZus{}interest}\PY{p}{(}\PY{n}{A}\PY{p}{,} \PY{n}{r}\PY{p}{,} \PY{n}{n}\PY{p}{)}\PY{p}{:}
            \PY{l+s+sd}{\PYZsq{}\PYZsq{}\PYZsq{}}
        \PY{l+s+sd}{    参数 A:本金}
        \PY{l+s+sd}{    参数 r:年利率}
        \PY{l+s+sd}{    参数 n:贷款年数}
        \PY{l+s+sd}{    输出 x:包含了月份,每月本金,每月偿还本金,每月偿还利息,每月偿还总金额的数据集}
        \PY{l+s+sd}{    \PYZsq{}\PYZsq{}\PYZsq{}}
            \PY{c+c1}{\PYZsh{} 计算月利率}
            \PY{n}{i} \PY{o}{=} \PY{n}{r} \PY{o}{/} \PY{l+m+mi}{12}
            \PY{c+c1}{\PYZsh{} 计算总的贷款月份}
            \PY{n}{m} \PY{o}{=} \PY{l+m+mi}{12} \PY{o}{*} \PY{n}{n}
            \PY{c+c1}{\PYZsh{} 类比一个月付息债券,即可直接计算出每月的付款额}
            \PY{n}{monthly\PYZus{}pay} \PY{o}{=} \PY{n}{A} \PY{o}{*} \PY{n}{i} \PY{o}{*} \PY{p}{(}\PY{l+m+mi}{1} \PY{o}{+} \PY{n}{i}\PY{p}{)} \PY{o}{*}\PY{o}{*} \PY{n}{m} \PY{o}{/} \PY{p}{(}\PY{p}{(}\PY{l+m+mi}{1} \PY{o}{+} \PY{n}{i}\PY{p}{)} \PY{o}{*}\PY{o}{*} \PY{n}{m} \PY{o}{\PYZhy{}} \PY{l+m+mi}{1}\PY{p}{)}
            \PY{c+c1}{\PYZsh{} 构建一个空的DataFrame用于储存数据}
            \PY{n}{x} \PY{o}{=} \PY{n}{pd}\PY{o}{.}\PY{n}{DataFrame}\PY{p}{(}\PY{p}{)}
            \PY{n}{x}\PY{p}{[}\PY{l+s+s1}{\PYZsq{}}\PY{l+s+s1}{月份}\PY{l+s+s1}{\PYZsq{}}\PY{p}{]} \PY{o}{=} \PY{p}{[}\PY{n}{j} \PY{k}{for} \PY{n}{j} \PY{o+ow}{in} \PY{n+nb}{range}\PY{p}{(}\PY{l+m+mi}{1}\PY{p}{,} \PY{n}{m} \PY{o}{+} \PY{l+m+mi}{1}\PY{p}{)}\PY{p}{]}
            \PY{n}{x}\PY{p}{[}\PY{l+s+s1}{\PYZsq{}}\PY{l+s+s1}{每月本金}\PY{l+s+s1}{\PYZsq{}}\PY{p}{]} \PY{o}{=} \PY{n}{np}\PY{o}{.}\PY{n}{nan}
            \PY{n}{x}\PY{p}{[}\PY{l+s+s1}{\PYZsq{}}\PY{l+s+s1}{每月偿还本金}\PY{l+s+s1}{\PYZsq{}}\PY{p}{]} \PY{o}{=} \PY{n}{np}\PY{o}{.}\PY{n}{nan}
            \PY{n}{x}\PY{p}{[}\PY{l+s+s1}{\PYZsq{}}\PY{l+s+s1}{每月偿还利息}\PY{l+s+s1}{\PYZsq{}}\PY{p}{]} \PY{o}{=} \PY{n}{np}\PY{o}{.}\PY{n}{nan}
            \PY{n}{x}\PY{p}{[}\PY{l+s+s1}{\PYZsq{}}\PY{l+s+s1}{每月偿还总金额}\PY{l+s+s1}{\PYZsq{}}\PY{p}{]} \PY{o}{=} \PY{p}{[}\PY{n}{monthly\PYZus{}pay} \PY{k}{for} \PY{n}{j} \PY{o+ow}{in} \PY{n+nb}{range}\PY{p}{(}\PY{n}{m}\PY{p}{)}\PY{p}{]}
            \PY{k}{for} \PY{n}{j} \PY{o+ow}{in} \PY{n+nb}{range}\PY{p}{(}\PY{n+nb}{len}\PY{p}{(}\PY{n}{x}\PY{p}{)}\PY{p}{)}\PY{p}{:}
                \PY{c+c1}{\PYZsh{} 计算处每月的本金额度}
                \PY{n}{x}\PY{o}{.}\PY{n}{iloc}\PY{p}{[}\PY{n}{j}\PY{p}{,} \PY{l+m+mi}{1}\PY{p}{]} \PY{o}{=} \PY{n}{A} \PY{o}{*} \PY{p}{(}\PY{l+m+mi}{1} \PY{o}{+} \PY{n}{i}\PY{p}{)} \PY{o}{*}\PY{o}{*} \PY{n}{j} \PY{o}{\PYZhy{}} \PY{n}{monthly\PYZus{}pay} \PY{o}{*} \PY{p}{(}\PY{p}{(}\PY{l+m+mi}{1} \PY{o}{+} \PY{n}{i}\PY{p}{)} \PY{o}{*}\PY{o}{*} \PY{n}{j} \PY{o}{\PYZhy{}} \PY{l+m+mi}{1}\PY{p}{)} \PY{o}{/} \PY{n}{i}
                \PY{c+c1}{\PYZsh{} 每月应付利息等于当月的本金额乘以月利率}
                \PY{n}{x}\PY{o}{.}\PY{n}{iloc}\PY{p}{[}\PY{n}{j}\PY{p}{,} \PY{l+m+mi}{3}\PY{p}{]} \PY{o}{=} \PY{n}{x}\PY{o}{.}\PY{n}{iloc}\PY{p}{[}\PY{n}{j}\PY{p}{,} \PY{l+m+mi}{1}\PY{p}{]} \PY{o}{*} \PY{n}{i}
                \PY{c+c1}{\PYZsh{} 由于月付本息和相等,减去利息即可得到偿还本金额度}
                \PY{n}{x}\PY{o}{.}\PY{n}{iloc}\PY{p}{[}\PY{n}{j}\PY{p}{,} \PY{l+m+mi}{2}\PY{p}{]} \PY{o}{=} \PY{n}{x}\PY{o}{.}\PY{n}{iloc}\PY{p}{[}\PY{n}{j}\PY{p}{,} \PY{l+m+mi}{4}\PY{p}{]} \PY{o}{\PYZhy{}} \PY{n}{x}\PY{o}{.}\PY{n}{iloc}\PY{p}{[}\PY{n}{j}\PY{p}{,} \PY{l+m+mi}{3}\PY{p}{]}
            \PY{k}{return} \PY{n}{x}
\end{Verbatim}


    \subsection{等额本息法例子}\label{ux7b49ux989dux672cux606fux6cd5ux4f8bux5b50}

    \begin{Verbatim}[commandchars=\\\{\}]
{\color{incolor}In [{\color{incolor}6}]:} \PY{n}{data} \PY{o}{=} \PY{n}{same\PYZus{}interest}\PY{p}{(}\PY{l+m+mi}{300000}\PY{p}{,} \PY{l+m+mf}{0.05}\PY{p}{,} \PY{l+m+mi}{15}\PY{p}{)}
        \PY{c+c1}{\PYZsh{} 选择画出两条曲线,即每月偿还本金和每月偿还利息}
        \PY{n}{plt}\PY{o}{.}\PY{n}{plot}\PY{p}{(}\PY{n}{data}\PY{p}{[}\PY{l+s+s1}{\PYZsq{}}\PY{l+s+s1}{月份}\PY{l+s+s1}{\PYZsq{}}\PY{p}{]}\PY{p}{,} \PY{n}{data}\PY{p}{[}\PY{l+s+s1}{\PYZsq{}}\PY{l+s+s1}{每月偿还本金}\PY{l+s+s1}{\PYZsq{}}\PY{p}{]}\PY{p}{,} \PY{n}{color}\PY{o}{=}\PY{l+s+s1}{\PYZsq{}}\PY{l+s+s1}{blue}\PY{l+s+s1}{\PYZsq{}}\PY{p}{,} \PYZbs{}
                     \PY{n}{linewidth}\PY{o}{=}\PY{l+m+mf}{2.5}\PY{p}{,} \PY{n}{label}\PY{o}{=}\PY{l+s+s1}{\PYZsq{}}\PY{l+s+s1}{Monthly Captial}\PY{l+s+s1}{\PYZsq{}}\PY{p}{)}
        \PY{n}{plt}\PY{o}{.}\PY{n}{plot}\PY{p}{(}\PY{n}{data}\PY{p}{[}\PY{l+s+s1}{\PYZsq{}}\PY{l+s+s1}{月份}\PY{l+s+s1}{\PYZsq{}}\PY{p}{]}\PY{p}{,} \PY{n}{data}\PY{p}{[}\PY{l+s+s1}{\PYZsq{}}\PY{l+s+s1}{每月偿还利息}\PY{l+s+s1}{\PYZsq{}}\PY{p}{]}\PY{p}{,} \PY{n}{color}\PY{o}{=}\PY{l+s+s1}{\PYZsq{}}\PY{l+s+s1}{red}\PY{l+s+s1}{\PYZsq{}}\PY{p}{,} \PYZbs{}
                     \PY{n}{linewidth}\PY{o}{=}\PY{l+m+mf}{2.5}\PY{p}{,} \PY{n}{label}\PY{o}{=}\PY{l+s+s1}{\PYZsq{}}\PY{l+s+s1}{Monthly Interest}\PY{l+s+s1}{\PYZsq{}}\PY{p}{)}
        \PY{c+c1}{\PYZsh{} 添加标题}
        \PY{n}{plt}\PY{o}{.}\PY{n}{title}\PY{p}{(}\PY{l+s+s1}{\PYZsq{}}\PY{l+s+s1}{Monthly Pay}\PY{l+s+s1}{\PYZsq{}}\PY{p}{)}
        \PY{n}{plt}\PY{o}{.}\PY{n}{xlabel}\PY{p}{(}\PY{l+s+s1}{\PYZsq{}}\PY{l+s+s1}{month}\PY{l+s+s1}{\PYZsq{}}\PY{p}{)}
        \PY{n}{plt}\PY{o}{.}\PY{n}{ylabel}\PY{p}{(}\PY{l+s+s1}{\PYZsq{}}\PY{l+s+s1}{capital or interest}\PY{l+s+s1}{\PYZsq{}}\PY{p}{)}
        \PY{c+c1}{\PYZsh{} 添加图例}
        \PY{n}{plt}\PY{o}{.}\PY{n}{legend}\PY{p}{(}\PY{p}{)}
        \PY{n}{plt}\PY{o}{.}\PY{n}{show}\PY{p}{(}\PY{p}{)}
\end{Verbatim}


    \begin{center}
    \adjustimage{max size={0.9\linewidth}{0.9\paperheight}}{output_13_0.png}
    \end{center}
    { \hspace*{\fill} \\}
    
    \section{债券的凸性}\label{ux503aux5238ux7684ux51f8ux6027}

    \subsection{债券的凸性计算公式}\label{ux503aux5238ux7684ux51f8ux6027ux8ba1ux7b97ux516cux5f0f}

\begin{itemize}
\tightlist
\item
  凸性的本质其实就是债券价格对市场利率的二阶导数
\item
  \(c = \frac{1}{p}\frac{d^2p}{dp^2}=\frac{1}{p(1+i)^2}[\sum_{t = t_1}^Tt(1+t)*\frac{coupon_t}{(1+i)^t}]\)
\item
  经过n次循环即可求出结果
\end{itemize}

    \begin{Verbatim}[commandchars=\\\{\}]
{\color{incolor}In [{\color{incolor}7}]:} \PY{k}{def} \PY{n+nf}{convex}\PY{p}{(}\PY{n}{A}\PY{p}{,} \PY{n}{c}\PY{p}{,} \PY{n}{n}\PY{p}{,} \PY{n}{i}\PY{p}{)}\PY{p}{:}
            \PY{l+s+sd}{\PYZsq{}\PYZsq{}\PYZsq{}}
        \PY{l+s+sd}{        参数 A:票面价值}
        \PY{l+s+sd}{        参数 c:票面利率}
        \PY{l+s+sd}{        参数 n:债券期限}
        \PY{l+s+sd}{        参数 i:市场利率}
        \PY{l+s+sd}{    \PYZsq{}\PYZsq{}\PYZsq{}}
            \PY{c+c1}{\PYZsh{} 计算出息票值}
            \PY{n}{c} \PY{o}{=} \PY{n}{A} \PY{o}{*} \PY{n}{c}
            \PY{c+c1}{\PYZsh{} 用P去存储每期支付的现值}
            \PY{n}{P} \PY{o}{=} \PY{l+m+mi}{0}
            \PY{c+c1}{\PYZsh{} 用X去储存前n\PYZhy{}1次循环使用息票的过程量,与最后一期分开计算}
            \PY{n}{X} \PY{o}{=} \PY{l+m+mi}{0}
            \PY{k}{for} \PY{n}{j} \PY{o+ow}{in} \PY{n+nb}{range}\PY{p}{(}\PY{l+m+mi}{1}\PY{p}{,} \PY{n}{n}\PY{p}{)}\PY{p}{:}
                \PY{n}{P} \PY{o}{+}\PY{o}{=} \PY{n}{c} \PY{o}{/} \PY{p}{(}\PY{l+m+mi}{1} \PY{o}{+} \PY{n}{i}\PY{p}{)} \PY{o}{*}\PY{o}{*} \PY{n}{j}
                \PY{n}{X} \PY{o}{+}\PY{o}{=} \PY{n}{j} \PY{o}{*} \PY{p}{(}\PY{n}{j} \PY{o}{+} \PY{l+m+mi}{1}\PY{p}{)} \PY{o}{*} \PY{n}{c} \PY{o}{/} \PY{p}{(}\PY{p}{(}\PY{l+m+mi}{1} \PY{o}{+} \PY{n}{i}\PY{p}{)} \PY{o}{*}\PY{o}{*} \PY{n}{j}\PY{p}{)}
            \PY{c+c1}{\PYZsh{} 最后一期因为要加上债券票面价值,所以单独拿出来计算}
            \PY{n}{P} \PY{o}{+}\PY{o}{=} \PY{p}{(}\PY{n}{A} \PY{o}{+} \PY{n}{c}\PY{p}{)} \PY{o}{/} \PY{p}{(}\PY{l+m+mi}{1} \PY{o}{+} \PY{n}{i}\PY{p}{)} \PY{o}{*}\PY{o}{*} \PY{n}{n}
            \PY{n}{X} \PY{o}{+}\PY{o}{=} \PY{n}{n} \PY{o}{*} \PY{p}{(}\PY{n}{n} \PY{o}{+} \PY{l+m+mi}{1}\PY{p}{)} \PY{o}{*} \PY{p}{(}\PY{n}{A} \PY{o}{+} \PY{n}{c}\PY{p}{)} \PY{o}{/} \PY{p}{(}\PY{p}{(}\PY{l+m+mi}{1} \PY{o}{+} \PY{n}{i}\PY{p}{)} \PY{o}{*}\PY{o}{*} \PY{n}{n}\PY{p}{)}
            \PY{c+c1}{\PYZsh{} 根据凸性的计算公式来计算即convex = 1/(P*(1+i)\PYZca{}2) * X}
            \PY{n}{con} \PY{o}{=} \PY{l+m+mi}{1} \PY{o}{/} \PY{p}{(}\PY{n}{P} \PY{o}{*} \PY{p}{(}\PY{l+m+mi}{1} \PY{o}{+} \PY{n}{i}\PY{p}{)} \PY{o}{*}\PY{o}{*} \PY{l+m+mi}{2}\PY{p}{)} \PY{o}{*} \PY{n}{X}
            \PY{n+nb}{print}\PY{p}{(}\PY{l+s+s2}{\PYZdq{}}\PY{l+s+s2}{经计算,对一个票面价值为}\PY{l+s+si}{\PYZpc{}s}\PY{l+s+s2}{,票面利率为}\PY{l+s+si}{\PYZpc{}0.2f}\PY{l+s+si}{\PYZpc{}\PYZpc{}}\PY{l+s+s2}{,期限为}\PY{l+s+si}{\PYZpc{}s}\PY{l+s+s2}{年的债券,凸性为}\PY{l+s+si}{\PYZpc{}s}\PY{l+s+se}{\PYZbs{}n}\PY{l+s+s2}{\PYZdq{}}\PY{p}{)}
                  \PY{o}{\PYZpc{}}\PY{p}{(}\PY{n}{A}\PY{p}{,}\PY{n}{c}\PY{p}{,}\PY{n}{n}\PY{p}{,}\PY{n}{con}\PY{p}{)}\PY{p}{)}
            \PY{n+nb}{print}\PY{p}{(}\PY{l+s+s1}{\PYZsq{}\PYZsq{}\PYZsq{}}\PY{l+s+s1}{\PYZhy{}\PYZhy{}\PYZhy{}\PYZhy{}\PYZhy{}\PYZhy{}\PYZhy{}\PYZhy{}\PYZhy{}\PYZhy{}\PYZhy{}\PYZhy{}\PYZhy{}\PYZhy{}\PYZhy{}\PYZhy{}\PYZhy{}\PYZhy{}\PYZhy{}\PYZhy{}\PYZhy{}\PYZhy{}\PYZhy{}\PYZhy{}\PYZhy{}\PYZhy{}\PYZhy{}\PYZhy{}\PYZhy{}\PYZhy{}\PYZhy{}\PYZhy{}\PYZhy{}\PYZhy{}\PYZhy{}\PYZhy{}\PYZhy{}\PYZhy{}\PYZhy{}\PYZhy{}\PYZhy{}\PYZhy{}\PYZhy{}\PYZhy{}\PYZhy{}\PYZhy{}\PYZhy{}\PYZhy{}\PYZhy{}\PYZhy{}\PYZhy{}\PYZhy{}\PYZhy{}\PYZhy{}\PYZhy{}\PYZhy{}}
        \PY{l+s+s1}{                \PYZhy{}\PYZhy{}\PYZhy{}\PYZhy{}\PYZhy{}\PYZhy{}\PYZhy{}\PYZhy{}\PYZhy{}\PYZhy{}\PYZhy{}\PYZhy{}\PYZhy{}\PYZhy{}\PYZhy{}\PYZhy{}\PYZhy{}\PYZhy{}\PYZhy{}\PYZhy{}\PYZhy{}\PYZhy{}\PYZhy{}\PYZhy{}\PYZhy{}\PYZhy{}\PYZhy{}\PYZhy{}\PYZhy{}\PYZhy{}\PYZhy{}\PYZhy{}\PYZhy{}\PYZhy{}\PYZhy{}\PYZhy{}\PYZhy{}\PYZhy{}\PYZhy{}\PYZhy{}\PYZhy{}\PYZhy{}}\PY{l+s+s1}{\PYZsq{}\PYZsq{}\PYZsq{}}\PY{p}{)}
\end{Verbatim}


    \subsection{计算凸性例子}\label{ux8ba1ux7b97ux51f8ux6027ux4f8bux5b50}

    \begin{Verbatim}[commandchars=\\\{\}]
{\color{incolor}In [{\color{incolor}8}]:} \PY{n+nb}{print}\PY{p}{(}\PY{l+s+s1}{\PYZsq{}}\PY{l+s+s1}{一组例子如下:}\PY{l+s+se}{\PYZbs{}n}\PY{l+s+s1}{\PYZsq{}}\PY{p}{)}
        \PY{n}{convex}\PY{p}{(}\PY{l+m+mi}{100}\PY{p}{,} \PY{l+m+mf}{0.07}\PY{p}{,} \PY{l+m+mi}{6}\PY{p}{,} \PY{l+m+mf}{0.1}\PY{p}{)}
        \PY{n}{convex}\PY{p}{(}\PY{l+m+mi}{100}\PY{p}{,} \PY{l+m+mf}{0.09}\PY{p}{,} \PY{l+m+mi}{7}\PY{p}{,} \PY{l+m+mf}{0.1}\PY{p}{)}
        \PY{n}{convex}\PY{p}{(}\PY{l+m+mi}{100}\PY{p}{,} \PY{l+m+mf}{0.14}\PY{p}{,} \PY{l+m+mi}{9}\PY{p}{,} \PY{l+m+mf}{0.1}\PY{p}{)}
        \PY{n}{convex}\PY{p}{(}\PY{l+m+mi}{100}\PY{p}{,} \PY{l+m+mf}{0.16}\PY{p}{,} \PY{l+m+mi}{8}\PY{p}{,} \PY{l+m+mf}{0.1}\PY{p}{)}
        \PY{n}{convex}\PY{p}{(}\PY{l+m+mi}{100}\PY{p}{,} \PY{l+m+mf}{0.18}\PY{p}{,} \PY{l+m+mi}{9}\PY{p}{,} \PY{l+m+mf}{0.1}\PY{p}{)}
\end{Verbatim}


    \begin{Verbatim}[commandchars=\\\{\}]
一组例子如下:

经计算,对于一个票面价值为100,票面利率为7.00\%,期限为6年的债券,凸性为27.315321199080740
-------------------------------------------------------------------------------------------
经计算,对于一个票面价值为100,票面利率为9.00\%,期限为7年的债券,凸性为32.824594195581300
-------------------------------------------------------------------------------------------
经计算,对于一个票面价值为100,票面利率为14.00\%,期限为9年的债券,凸性为41.83973607681119
-------------------------------------------------------------------------------------------
经计算,对于一个票面价值为100,票面利率为16.00\%,期限为8年的债券,凸性为34.46203357957334
-------------------------------------------------------------------------------------------
经计算,对于一个票面价值为100,票面利率为18.00\%,期限为9年的债券,凸性为39.14042134957644
-------------------------------------------------------------------------------------------

    \end{Verbatim}

    \section{利率期限结构}\label{ux5229ux7387ux671fux9650ux7ed3ux6784}

    \subsection{利率期限结构计算公式}\label{ux5229ux7387ux671fux9650ux7ed3ux6784ux8ba1ux7b97ux516cux5f0f}

\begin{itemize}
\tightlist
\item
  利率期限结构是指在某一时点上,不同期限资金的收益率与到期期限之间的关系
\item
  对于操作而言,其实就是根据已知的即期利率,去计算更远期限的即期利率,是一个从前往后不断推进的过程
\item
  需要注意的是一年之内的所有债券的即期利率就是到期收益率,所以可以直接相等
\item
  为了方便起见我默认使用的债券数据都是一年付息两次,传入的数据集里默认已经有债券基本信息包括到期收益率
\item
  列举其中2步计算过程(因为半年和一年是直接相等,所以从1.5年开始计算):

  \begin{itemize}
  \tightlist
  \item
    \(r_1 = YTM_{0.5}\)
  \item
    \(r_2 = YTM_1\)
  \item
    \(par=\frac{coupon}{1+r_1/2}+\frac{coupon}{(1+r_2/2)^2}+\frac{coupon+par}{(1+r_3/2)^3}\)
  \item
    其中par为面值,coupon为第1.5年的现金流,带入\(r_1\)和\(r_2\)求解出\(r_{3}\)即为1.5年的即期利率
  \item
    \(par=\frac{coupon}{1+r_1/2}+\frac{coupon}{(1+r_2/2)^2}+\frac{coupon}{(1+r_3/2)^3}+\frac{coupon+par}{(1+r_4/2)^4}\)
  \item
    于是带入之前的r,求出第二年的即期利率即可
  \item
    重复上述过程,最后便能求得利率期限结构
  \end{itemize}
\end{itemize}

\begin{Verbatim}[commandchars=\\\{\}]
  {\color{incolor}In [{\color{incolor}9}]:} \PY{k}{def} \PY{n+nf}{IRR}\PY{p}{(}\PY{n}{data}\PY{p}{)}\PY{p}{:}
              \PY{n}{data}\PY{p}{[}\PY{l+s+s1}{\PYZsq{}}\PY{l+s+s1}{即期收益率}\PY{l+s+s1}{\PYZsq{}}\PY{p}{]} \PY{o}{=} \PY{n}{np}\PY{o}{.}\PY{n}{nan}
              \PY{c+c1}{\PYZsh{} 一年期内的到期利率即为即期利率}
              \PY{n}{data}\PY{o}{.}\PY{n}{iloc}\PY{p}{[}\PY{l+m+mi}{0}\PY{p}{,} \PY{l+m+mi}{3}\PY{p}{]} \PY{o}{=} \PY{n}{data}\PY{o}{.}\PY{n}{iloc}\PY{p}{[}\PY{l+m+mi}{0}\PY{p}{,} \PY{l+m+mi}{2}\PY{p}{]}
              \PY{n}{data}\PY{o}{.}\PY{n}{iloc}\PY{p}{[}\PY{l+m+mi}{1}\PY{p}{,} \PY{l+m+mi}{3}\PY{p}{]} \PY{o}{=} \PY{n}{data}\PY{o}{.}\PY{n}{iloc}\PY{p}{[}\PY{l+m+mi}{1}\PY{p}{,} \PY{l+m+mi}{2}\PY{p}{]}
              \PY{k}{for} \PY{n}{i} \PY{o+ow}{in} \PY{n+nb}{range}\PY{p}{(}\PY{l+m+mi}{2}\PY{p}{,} \PY{n+nb}{len}\PY{p}{(}\PY{n}{data}\PY{p}{)}\PY{p}{)}\PY{p}{:}
                  \PY{c+c1}{\PYZsh{} 将即期收益率设为x}
                  \PY{n}{x} \PY{o}{=} \PY{n}{symbols}\PY{p}{(}\PY{l+s+s1}{\PYZsq{}}\PY{l+s+s1}{x}\PY{l+s+s1}{\PYZsq{}}\PY{p}{)}
                  \PY{c+c1}{\PYZsh{} 计算息票值,因为百分数所以除以100}
                  \PY{n}{coupon} \PY{o}{=} \PY{n}{data}\PY{o}{.}\PY{n}{iloc}\PY{p}{[}\PY{n}{i}\PY{p}{,} \PY{l+m+mi}{0}\PY{p}{]} \PY{o}{*} \PY{n}{data}\PY{o}{.}\PY{n}{iloc}\PY{p}{[}\PY{n}{i}\PY{p}{,} \PY{l+m+mi}{1}\PY{p}{]} \PY{o}{/} \PY{l+m+mi}{100}
                  \PY{c+c1}{\PYZsh{} 用p来储存前面n\PYZhy{}1期的现金流贴现总和}
                  \PY{n}{p} \PY{o}{=} \PY{l+m+mi}{0}
                  \PY{k}{for} \PY{n}{j} \PY{o+ow}{in} \PY{n+nb}{range}\PY{p}{(}\PY{n}{i}\PY{p}{)}\PY{p}{:}
                      \PY{n}{p} \PY{o}{+}\PY{o}{=} \PY{n}{coupon} \PY{o}{/} \PY{p}{(}\PY{p}{(}\PY{l+m+mi}{1} \PY{o}{+} \PY{n}{data}\PY{o}{.}\PY{n}{iloc}\PY{p}{[}\PY{n}{j}\PY{p}{,} \PY{l+m+mi}{3}\PY{p}{]} \PY{o}{/} \PY{l+m+mi}{200}\PY{p}{)} \PY{o}{*}\PY{o}{*} \PY{p}{(}\PY{n}{j} \PY{o}{+} \PY{l+m+mi}{1}\PY{p}{)}\PY{p}{)}
                  \PY{c+c1}{\PYZsh{} 使用Sympy的solve函数直接接触即期利率}
                  \PY{n}{a} \PY{o}{=} \PY{n}{solve}\PY{p}{(}\PY{n}{data}\PY{o}{.}\PY{n}{iloc}\PY{p}{[}\PY{n}{i}\PY{p}{,} \PY{l+m+mi}{0}\PY{p}{]} \PY{o}{\PYZhy{}} \PY{n}{p} \PY{o}{\PYZhy{}} \PY{p}{(}\PY{p}{(}\PY{n}{data}\PY{o}{.}\PY{n}{iloc}\PY{p}{[}\PY{n}{i}\PY{p}{,} \PY{l+m+mi}{0}\PY{p}{]} \PYZbs{}
                                                    \PY{o}{+} \PY{n}{coupon}\PY{p}{)} \PY{o}{/} \PY{p}{(}\PY{p}{(}\PY{l+m+mi}{1} \PY{o}{+} \PY{n}{x} \PY{o}{/} \PY{l+m+mi}{2}\PY{p}{)} \PY{o}{*}\PY{o}{*} \PY{p}{(}\PY{n}{i} \PY{o}{+} \PY{l+m+mi}{1}\PY{p}{)}\PY{p}{)}\PY{p}{)}\PY{p}{,} \PY{n}{x}\PY{p}{)}
                  \PY{c+c1}{\PYZsh{} 因为会解出复数解,所以根据类型筛选数据后取其中最大的实数字作为即期利率}
                  \PY{n}{r} \PY{o}{=} \PY{n+nb}{max}\PY{p}{(}\PY{p}{[}\PY{n}{x} \PY{k}{for} \PY{n}{x} \PY{o+ow}{in} \PY{n}{a} \PY{k}{if} \PY{n+nb}{type}\PY{p}{(}\PY{n}{x}\PY{p}{)} \PY{o}{==} \PY{n}{Float}\PY{p}{]}\PY{p}{)}
                  \PY{c+c1}{\PYZsh{} 将半年的名义利率转换为一年的,如有需要可以更改为实际利率,但要与之前一起更改}
                  \PY{n}{data}\PY{o}{.}\PY{n}{iloc}\PY{p}{[}\PY{n}{i}\PY{p}{,} \PY{l+m+mi}{3}\PY{p}{]} \PY{o}{=} \PY{n}{r} \PY{o}{*} \PY{l+m+mi}{100}
              \PY{k}{return} \PY{n}{data}
\end{Verbatim}


    \subsection{利率期限结构简单例子}\label{ux5229ux7387ux671fux9650ux7ed3ux6784ux7b80ux5355ux4f8bux5b50}

    \begin{Verbatim}[commandchars=\\\{\}]
{\color{incolor}In [{\color{incolor}10}]:} \PY{c+c1}{\PYZsh{} 生成一组原始数据}
         \PY{n}{data} \PY{o}{=} \PY{n}{pd}\PY{o}{.}\PY{n}{DataFrame}\PY{p}{(}\PY{p}{)}
         \PY{n}{data}\PY{p}{[}\PY{l+s+s1}{\PYZsq{}}\PY{l+s+s1}{债券期限}\PY{l+s+s1}{\PYZsq{}}\PY{p}{]} \PY{o}{=} \PY{p}{[}\PY{l+m+mf}{0.5}\PY{p}{,} \PY{l+m+mi}{1}\PY{p}{,} \PY{l+m+mf}{1.5}\PY{p}{,} \PY{l+m+mi}{2}\PY{p}{,} \PY{l+m+mf}{2.5}\PY{p}{,} \PY{l+m+mi}{3}\PY{p}{]}
         \PY{n}{x} \PY{o}{=} \PY{n}{data}\PY{p}{[}\PY{l+s+s1}{\PYZsq{}}\PY{l+s+s1}{债券期限}\PY{l+s+s1}{\PYZsq{}}\PY{p}{]} \PY{o}{=} \PY{p}{[}\PY{l+m+mf}{0.5}\PY{p}{,} \PY{l+m+mi}{1}\PY{p}{,} \PY{l+m+mf}{1.5}\PY{p}{,} \PY{l+m+mi}{2}\PY{p}{,} \PY{l+m+mf}{2.5}\PY{p}{,} \PY{l+m+mi}{3}\PY{p}{]}
         \PY{n}{data}\PY{p}{[}\PY{l+s+s1}{\PYZsq{}}\PY{l+s+s1}{债券初始面值}\PY{l+s+s1}{\PYZsq{}}\PY{p}{]} \PY{o}{=} \PY{p}{[}\PY{l+m+mi}{100}\PY{p}{,} \PY{l+m+mi}{100}\PY{p}{,} \PY{l+m+mi}{100}\PY{p}{,} \PY{l+m+mi}{100}\PY{p}{,} \PY{l+m+mi}{100}\PY{p}{,} \PY{l+m+mi}{100}\PY{p}{]}
         \PY{n}{data}\PY{p}{[}\PY{l+s+s1}{\PYZsq{}}\PY{l+s+s1}{票面利率}\PY{l+s+s1}{\PYZsq{}}\PY{p}{]} \PY{o}{=} \PY{p}{[}\PY{l+m+mf}{1.9}\PY{p}{,} \PY{l+m+mf}{2.1}\PY{p}{,} \PY{l+m+mf}{2.3}\PY{p}{,} \PY{l+m+mf}{2.8}\PY{p}{,} \PY{l+m+mf}{3.7}\PY{p}{,} \PY{l+m+mf}{5.5}\PY{p}{]}
         \PY{n}{data}\PY{p}{[}\PY{l+s+s1}{\PYZsq{}}\PY{l+s+s1}{到期收益率}\PY{l+s+s1}{\PYZsq{}}\PY{p}{]} \PY{o}{=} \PY{p}{[}\PY{l+m+mf}{4.0}\PY{p}{,} \PY{l+m+mf}{4.2}\PY{p}{,} \PY{l+m+mf}{4.6}\PY{p}{,} \PY{l+m+mf}{5.6}\PY{p}{,} \PY{l+m+mf}{5.8}\PY{p}{,} \PY{l+m+mf}{7.0}\PY{p}{]}
         \PY{n}{data}\PY{o}{.}\PY{n}{set\PYZus{}index}\PY{p}{(}\PY{l+s+s1}{\PYZsq{}}\PY{l+s+s1}{债券期限}\PY{l+s+s1}{\PYZsq{}}\PY{p}{,} \PY{n}{inplace}\PY{o}{=}\PY{k+kc}{True}\PY{p}{)}
         \PY{c+c1}{\PYZsh{} 计算这组数据的利率期限结构}
         \PY{n}{result} \PY{o}{=} \PY{n}{IRR}\PY{p}{(}\PY{n}{data}\PY{p}{)}
         \PY{n}{result}
\end{Verbatim}


\begin{Verbatim}[commandchars=\\\{\}]
  {\color{outcolor}Out[{\color{outcolor}10}]:}       债券初始面值  票面利率(%)  到期收益率(%)       即期收益率(%)
         债券期限                                       
           0.5      100          1.9          4.0               4.00000000000000
           1.0      100          2.1          4.2               4.20000000000000
           1.5      100          2.3          4.6               4.61087216259755
           2.0      100          2.8          5.6               5.65329297457169
           2.5      100          3.7          5.8               7.60002114620885
           3.0      100          5.5          7.0               11.8398918196558
  \end{Verbatim}
            
    \subsection{利率期限结构图}\label{ux5229ux7387ux671fux9650ux7ed3ux6784ux56fe}

    \begin{Verbatim}[commandchars=\\\{\}]
{\color{incolor}In [{\color{incolor}11}]:} \PY{n}{plt}\PY{o}{.}\PY{n}{plot}\PY{p}{(}\PY{n}{x}\PY{p}{,}\PY{n}{data}\PY{p}{[}\PY{l+s+s1}{\PYZsq{}}\PY{l+s+s1}{即期收益率}\PY{l+s+s1}{\PYZsq{}}\PY{p}{]}\PY{p}{)}
         \PY{n}{plt}\PY{o}{.}\PY{n}{title}\PY{p}{(}\PY{l+s+s1}{\PYZsq{}}\PY{l+s+s1}{Term Structure of Interest Rates}\PY{l+s+s1}{\PYZsq{}}\PY{p}{)}
         \PY{n}{plt}\PY{o}{.}\PY{n}{xlabel}\PY{p}{(}\PY{l+s+s1}{\PYZsq{}}\PY{l+s+s1}{time}\PY{l+s+s1}{\PYZsq{}}\PY{p}{)}
         \PY{n}{plt}\PY{o}{.}\PY{n}{ylabel}\PY{p}{(}\PY{l+s+s1}{\PYZsq{}}\PY{l+s+s1}{sight rate}\PY{l+s+s1}{\PYZsq{}}\PY{p}{)}
         \PY{n}{plt}\PY{o}{.}\PY{n}{show}\PY{p}{(}\PY{p}{)}
\end{Verbatim}


    \begin{center}
    \adjustimage{max size={0.9\linewidth}{0.9\paperheight}}{output_25_0.png}
    \end{center}
    { \hspace*{\fill} \\}
    
    通常如果计算市场的利率期限结构,我会选择使用 MATLAB 金融工具箱中的
zbtyield 函数,较为方便


    % Add a bibliography block to the postdoc
    
    
    
    \end{document}
